\documentclass[11pt]{article}

% This is a helpful package that puts math inside length specifications
\usepackage{calc}

% Layout: Puts the section titles on left side of page
\reversemarginpar

%
%         PAPER SIZE, PAGE NUMBER, AND DOCUMENT LAYOUT NOTES:
%
% The next \usepackage line changes the layout for CV style section
% headings as marginal notes. It also sets up the paper size as either
% letter or A4. By default, letter was used. If A4 paper is desired,
% comment out the letterpaper lines and uncomment the a4paper lines.
%
% As you can see, the margin widths and section title widths can be
% easily adjusted.
%
% ALSO: Notice that the includefoot option can be commented OUT in order
% to put the PAGE NUMBER *IN* the bottom margin. This will make the
% effective text area larger.
%
% IF YOU WISH TO REMOVE THE ``of LASTPAGE'' next to each page number,
% see the note about the +LP and -LP lines below. Comment out the +LP
% and uncomment the -LP.
%
% IF YOU WISH TO REMOVE PAGE NUMBERS, be sure that the includefoot line
% is uncommented and ALSO uncomment the \pagestyle{empty} a few lines
% below.
%

%% Use these lines for letter-sized paper
%% \usepackage[paper=letterpaper,
%%             %includefoot, % Uncomment to put page number above margin
%%             marginparwidth=1in,     % Length of section titles
%%             marginparsep=.05in,       % Space between titles and text
%%             margin=0.5in,               % 1 inch margins
%%             includemp]{geometry}

%% Use these lines for A4-sized paper
\usepackage[paper=a4paper,
           %includefoot, % Uncomment to put page number above margin
           marginparwidth=30.5mm,    % Length of section titles
           marginparsep=1.5mm,       % Space between titles and text
           margin=20mm,              % 25mm margins
           includemp]{geometry}

%% More layout: Get rid of indenting throughout entire document
\setlength{\parindent}{0in}

%% This gives us fun enumeration environments. compactitem will be nice.
\usepackage{paralist}

%% Reference the last page in the page number
%
% NOTE: comment the +LP line and uncomment the -LP line to have page
%       numbers without the ``of ##'' last page reference)
%
% NOTE: uncomment the \pagestyle{empty} line to get rid of all page
%       numbers (make sure includefoot is commented out above)
%
\usepackage{fancyhdr,lastpage}
\pagestyle{fancy}
%\pagestyle{empty}      % Uncomment this to get rid of page numbers
\fancyhf{}\renewcommand{\headrulewidth}{0pt}
\fancyfootoffset{\marginparsep+\marginparwidth}
\newlength{\footpageshift}
\setlength{\footpageshift}
          {0.5\textwidth+0.5\marginparsep+0.5\marginparwidth-2in}
\lfoot{\hspace{\footpageshift}%
       \parbox{4in}{\, \hfill %
                    \arabic{page} of \protect\pageref*{LastPage} % +LP
%                    \arabic{page}                               % -LP
                    \hfill \,}}

% Finally, give us PDF bookmarks
\usepackage{color,hyperref}
\definecolor{darkblue}{rgb}{0.0,0.0,0.3}
\hypersetup{colorlinks,breaklinks,
            linkcolor=darkblue,urlcolor=darkblue,
            anchorcolor=darkblue,citecolor=darkblue}

%\usepackage{siunitx}
%\DeclareSIUnit\TeV{TeV}
\usepackage{amsmath}
\usepackage{multicol}

%%%%%%%%%%%%%%%%%%%%%%%% End Document Setup %%%%%%%%%%%%%%%%%%%%%%%%%%%%


%%%%%%%%%%%%%%%%%%%%%%%%%%% Helper Commands %%%%%%%%%%%%%%%%%%%%%%%%%%%%

% The title (name) with a horizontal rule under it
%
% Usage: \makeheading{name}
%
% Place at top of document. It should be the first thing.
\newcommand{\makeheading}[1]%
        {\hspace*{-\marginparsep minus \marginparwidth}%
         \begin{minipage}[t]{\textwidth+\marginparwidth+\marginparsep}%
                {\Large \bfseries #1}\\[-0.15\baselineskip]%
                 \rule{\columnwidth}{1pt}%
         \end{minipage}}

% The section headings
%
% Usage: \section{section name}
%
% Follow this section IMMEDIATELY with the first line of the section
% text. Do not put whitespace in between. That is, do this:
%
%       \section{My Information}
%       Here is my information.
%
% and NOT this:
%
%       \section{My Information}
%
%       Here is my information.
%
% Otherwise the top of the section header will not line up with the top
% of the section. Of course, using a single comment character (%) on
% empty lines allows for the function of the first example with the
% readability of the second example.
\renewcommand{\section}[2]%
        {\pagebreak[2]\vspace{1.3\baselineskip}%
         \phantomsection\addcontentsline{toc}{section}{#1}%
         \hspace{0in}%
         \marginpar{
         \raggedright \scshape #1}#2}

% An itemize-style list with lots of space between items
\newenvironment{outerlist}[1][\enskip\textbullet]%
        {\begin{itemize}[#1]}{\end{itemize}%
         \vspace{-.6\baselineskip}}

% An environment IDENTICAL to outerlist that has better pre-list spacing
% when used as the first thing in a \section
\newenvironment{lonelist}[1][\enskip\textbullet]%
        {\vspace{-\baselineskip}\begin{list}{#1}{%
        \setlength{\partopsep}{0pt}%
        \setlength{\topsep}{0pt}}}
        {\end{list}\vspace{-.6\baselineskip}}

% An itemize-style list with little space between items
\newenvironment{innerlist}[1][\enskip\textbullet]%
        {\begin{compactitem}[#1]}{\end{compactitem}}

% To add some paragraph space between lines.
% This also tells LaTeX to preferably break a page on one of these gaps
% if there is a needed pagebreak nearby.
\newcommand{\blankline}{\quad\pagebreak[2]}

%%%%%%%%%%%%%%%%%%%%%%%% End Helper Commands %%%%%%%%%%%%%%%%%%%%%%%%%%%

%%%%%%%%%%%%%%%%%%%%%%%%% Begin CV Document %%%%%%%%%%%%%%%%%%%%%%%%%%%%

\begin{document}
\makeheading{Henrik Qvigstad}

\section{Contact Information}
%
% NOTE: Mind where the & separators and \\ breaks are in the following
%       table.
%
% ALSO: \rcollength is the width of the right column of the table
%       (adjust it to your liking; default is 1.85in).
%
\newlength{\rcollength}\setlength{\rcollength}{7cm}%
%
\begin{tabular}[t]{@{}p{\textwidth-\rcollength}p{\rcollength}}
%% \href{http://www.mn.uio.no/fysikk/english/}%
%%      {Department of Physics}
Henrik Qvigstad            & \textit{Phone:} +47 97606794 \\
%% \href{http://www.uio.no/english/}
%%      {University of Oslo}
Nils Huus Gate 18          &  \textit{Fax:} +47 22856422\\
0482 Oslo               & \textit{E-mail:}
                             \href{mailto:henrik.qvigstad@gmail.com}
                                  {henrik.qvigstad@gmail.com}\\
Norway
\end{tabular}

\section{Citizenship}
%
Norwegian

\vspace{-0.5cm}

\section{Summary}
%
Data scientist, experienced with:
\begin{multicols}{2}
\begin{itemize}
\item Javascript
\item C/C++
\item Data Analysis
\item {\small Algorithm Testing and Evaluating}
\item Large Scale Computing (GRID)
\item {\small Electronics Commissioning/Maintenance.}
\end{itemize}
\end{multicols}

\vspace{-0.5cm}

\section{Interests}
%
Data Analysis, Applied Statistics, Parameter Optimization,
Information Retrieval
%% , Artificial Intelligence

%
\section{Work Experience}

\href{https://www.opera.com/}{\textbf{Opera Software AS}}
Oslo, Norway
\begin{outerlist}
\item[] \textit{Javascript Developer}
  \hfill June 2015 to November 2016
  \begin{innerlist}
      \footnotesize
    \item UI Developer for Opera's desktop browser.
      \begin{innerlist}
        \item Implement features, AB tests, and conduct Experimental Development.
        \item Made use of using chrome's extension api, as well as internal api's.
        \item Develop using cutting edge web standards,
          including features of ES6, HTML5, and CSS4.
      \end{innerlist}
  \end{innerlist}
\end{outerlist}
\blankline

\href{http://www.uib.no/en}{\textbf{University of Oslo}}
Oslo, Norway \& Geneva, Switzerland
\begin{outerlist}
\item[] \textit{Researcher, PhD Candidate}
  \hfill 2010 to 2015
  \begin{innerlist}
    \footnotesize
  \item Main thesis work is on the analysis of data produced by ALICE.
    \begin{innerlist}
    \item Analysis is done through the development and use of the
      C++ software library:
      \href{http://aliweb.cern.ch/Offline/AliRoot/Manual.html}{AliROOT}.
    \item Large Scale Analysis is done using The
      \href{http://wlcg.web.cern.ch/}{WorldWide LHC Computing Grid GRID}.
    \item I serve as an Train Operator for the \href{alien2.cern.ch}{AliEn}
      \href{http://aliweb.cern.ch/Offline/Activities/Analysis/AnalysisFramework/index.html}{Analysis Framework} LEGO Train System, an
      \href{http://alimonitor.cern.ch/map.jsp}{front end/interface}
      for the use of GRID by the Analysis groups of ALICE.
    \end{innerlist}
  \item I served as PHOS Run Coordinator:
    \begin{innerlist}
    \item which required a permanent residence at CERN, Switzerland,
      June 11' to Feb. 13'
    \item coordinated and participated in the Commissioning,
      Development and Maintenance of PHOS and its monitoring and control systems;
    \item as expert on-call for PHOS, as well as Shift-Leader for ALICE,
      effectively controlling and monitoring the billion dollar experiment.
    \end{innerlist}
  \item Teaching Assistant, in \href{http://www.uio.no/studier/emner/matnat/fys/FYS2160/index-eng.html}{Thermodynamics}.
  \end{innerlist}
\end{outerlist}

\blankline
%% \newpage

\href{http://www.uib.no/en}{\textbf{University of Bergen}},
Geneva, Switzerland \& Bergen, Norway
\begin{outerlist}
\item[] \textit{Research Assistant}
  \hfill 2009 to 2010
  \begin{innerlist}
    \footnotesize
  %% \item Continued service work done during master thesis;
  %%   commissioning of PHOS, The Photon Spectrometer.
  \item Developed software for the numerical calculation and imaging of
    the detector status.
  \item Worked directly with the PHOS commissioning team at ALICE, CERN.
  \end{innerlist}
\item[] \textit{Teaching Assistant}
  \hfill Fall '07
  \begin{innerlist}
    \footnotesize
  \item \href{http://www.uib.no/course/STAT101#course}{Elementary Statistics}
  \item Held lab sessions introducing students to the use of S-Plus statistics
    software, and graded assignments.
  \end{innerlist}
\end{outerlist}

\blankline

%% \href{http://www.vilvite.no/english}{\textbf{VilVite, Bergen Science Center}},
%% Bergen, Norway
%% \begin{outerlist}
%% \item[] \textit{Science Host}
%%   \hfill Fall '07 to Fall '08
%%   \begin{innerlist}
%%     \footnotesize
%%   %% \item Hosted visits, requiring good interpersonal skills and the ability to
%%   %%   explain the science behind display pieces to the general public.
%%   \item Host and Content Development,
%%   \item including the building a \href{http://en.wikipedia.org/wiki/Cloud_chamber}{Cloud Chamber}, and particle detector.
%%   \end{innerlist}
%% \end{outerlist}
%%
%% \blankline

%% \textbf{The Norwegian Army},
%% Troms, Norway
%% \begin{outerlist}
%% \item[] \textit{Conscript}, corporal
%%   \hfill June 03' to June 04'
%%   %% \begin{innerlist}
%%   %% \item Corporal
%%   %% \end{innerlist}
%% \end{outerlist}



\section{Education}
%
\href{http://www.uio.no/english/}{\textbf{University of Oslo}},
Oslo, Norway
\begin{outerlist}

\item[] \textbf{Ph.D.},
  \href{http://www.mn.uio.no/fysikk/english/}
       {Department of Physics}, expected graduation date: 2015
        \begin{innerlist}
          \footnotesize
        \item Thesis Topic: ``Production of neutral particles in ALICE using PHOS, The Photon Spectrometer.''
        %% \item Advisor:
        %%       \href{http://www.mn.uio.no/fysikk/english/people/aca/trine/}
        %%            {Trine S. Tveter}
        %% \item Area of Study:
        %%                 Relativistic Heavy Ion Collisions
        \end{innerlist}

\end{outerlist}

%\vspace{\baselineskip}
%\bigskip
\blankline

\href{http://www.uib.no/en}{\textbf{University of Bergen}},
Bergen, Norway
\begin{outerlist}

\item[] \textbf{M.S.},
  \href{http://www.uib.no/ift/en}
       {Department of Physics and Technology}, 2009
        \begin{innerlist}
          \footnotesize
        \item Thesis Topic: ``Calibration of ALICE Photon Spectrometer
          using $\pi^0\rightarrow\gamma\gamma$ decay.''
        %% \item Advisor:
        %%       \href{http://web.ift.uib.no/~joakim/}{Joakim Nystrand}
        %% \item Area of Study:
        %%       \href{http://www.uib.no/programmeoption/MAMN-FYKJR#description}
        %%            {Nuclear Physics}
        \end{innerlist}

\item[] \href{http://www.uib.no/studyprogramme/BAMN-PHYS}
        {\textbf{B.S.}},
        \href{http://www.uib.no/matnat/en}
             {Faculty of Mathematics and Natural Sciences}, 2007
        \begin{innerlist}
          \footnotesize
        \item Specialization (Major/minor) within Particle and Nuclear Physics,
          computational mathematics, and programming.
        \end{innerlist}

\end{outerlist}



%% \section{General Skills}

%% Is experienced working as part of larger groups with long term goals,
%% including experience as part of lower management.

%% \blankline

%% Driving: Norwegian class-B license.

%% \blankline

%% \hspace{-3mm}
%% \begin{tabular*}{1.0\linewidth}{lll}
%% \emph{Language:}  & Norwegian: & First Language \\
%%             & English: & Fluent
%%               , some experience in Technical Writing (Documentation)
%% \end{tabular*}

\section{Language}
%
%\hspace{-3.5mm}
 Norwegian: First Language \newline
 English:  Fluent, some experience in Technical Writing (Documentation)

%% \begin{tabular*}{1.0\linewidth}{ll}
%%  Norwegian: & First Language \\
%%  English: & Fluent
%%               , some experience in Technical Writing (Documentation)
%% \end{tabular*}

\section{Technical Skills}
%
Extensive experience with Data Analysis, as well as maintaining and developing high level electronics control.


\blankline

\emph{Programming:} Javascript, C/C++, UNIX shell scripting, Java, Python, and cmake.

\blankline

\emph{Libraries:} Qt, SWING, ROOT, AliRoot, AliAnalysis.

\blankline

\emph{Applications:} Git, autotools, S-Plus,
              \TeX{}, \LaTeX{}, B\textsc{ib}\TeX{}, alien2/aliensh,
        \href{http://www.ni.com/}{LabVIEW},
        \href{http://www.etm.at/index_e.asp?}{ETM PVSS-II},
        \href{http://www.libreoffice.org/}{Libre Office},
        and other common productivity packages for Windows and
        Linux platforms

\blankline

\emph{Operating Systems:} Windows 7, Microsoft Windows XP/2000, BSD,
and extensive experience using Linux professionally: mostly RedHat-family,
but also including Debian (Ubuntu) and pacman based distributions.

\blankline

\emph{Mathematical Expertise:}
\emph{equivalent of Mathematics B.S. Graduate with experience in the following subjects:}
General Calculus, Linear Algebra, Differential Equations, Statistics,
and Numerical Analysis.



\section{Workshops}
%

\textbf{Programming of Multicore Systems}
\newline
\begin{tabular*}{\textwidth}{@{\extracolsep{\fill}} l r}
  International Research Training Group (IRTG) &  9-13. Feb. 2009 \\
  Bergen University College &
\end{tabular*}

\blankline

\textbf{Application of nuclear and particle physics methods to medical physics}
\newline
\begin{tabular*}{\textwidth}{@{\extracolsep{\fill}} l r}
  International Research Training Group (IRTG) & 2-5. May. 2011\\
  Bergen University College &
\end{tabular*}

\blankline

%% \textbf{First experience and results from LHC}
%% \newline
%% \begin{tabular*}{\textwidth}{@{\extracolsep{\fill}} l r}
%%   International Research Training Group (IRTG) & 10-14. May. 2010\\
%%   Heidelberg University &
%% \end{tabular*}

%% \blankline

%% \textbf{Measurements and sensor operation in harsh and remote environments}
%% \newline
%% \begin{tabular*}{\textwidth}{@{\extracolsep{\fill}} l r}
%%   International Research Training Group (IRTG) & 15-19. Oct. 2012\\
%%   Bergen University College &
%% \end{tabular*}

%% \blankline

%% \textbf{ALICE Offline Tutorial}
%% \newline
%% \begin{tabular*}{\textwidth}{@{\extracolsep{\fill}} l r}
%%   ALICE Experiment & 1-2. May. 2009 \\
%%   CERN &
%% \end{tabular*}

etc.


\section{Publications}
%

%% \begin{outerlist}
%% \item[] Neutral pion and $\eta$ meson production in proton–proton collisions at
%% $\sqrt{s}=0.9\ TeV$ and $\sqrt{s}=0.9\ TeV$,\href{http://dx.doi.org/10.1016/j.physletb.2012.09.015}{Physics Letters B}, 2012
%% \item[] Production of neutral pions and eta-mesons in pp collisions measured with ALICE,
%% \href{http://dx.doi.org/10.1088/0954-3899/38/12/124076}{Journal of Physics G: Nuclear and Particle Physics}, 2011
%% \item[] Master Thesis: Calibration of the ALICE Photon Spectrometer (PHOS) using the $\pi^0 \leftarrow \gamma\gamma$ decay, Universitetet i Bergen, 2009,
%% (\href{http://folk.uio.no/henrikq/docs/master-thesis.pdf}{pdf})
%% \end{outerlist}

For full list see \href{http://www.cristin.no/as/WebObjects/cristin.woa/wa/fres?sort=ar&la=no&action=sok&pnr=1710}{Cristin.no}
(in case of paper copy: \href{http://www.cristin.no/as/WebObjects/cristin.woa/wa/fres?sort=ar&la=no&action=sok&pnr=1710}{http://goo.gl/RFGCkO})





%% \blankline

%% \href{http://www.mathworks.com/products/matlab/}{\textsc{Matlab}}
%%         experience: linear algebra, Fourier transforms,
%%         nonlinear numerical methods, polynomials, statistics,
%%         visualization

%% \blankline

%% \href{http://www.mathworks.com/products/matlab/}{\textsc{Matlab}}
%%         toolboxes: communications, control system, filter
%%         design, genetic algorithm and direct search, signal processing,
%%         system identification


%% \section{Books in Preparation}
%% %
%% Pavlic, T.P., B.W.~Andrews, K.M.~Passino, and T.A.~Waite. Foraging
%% Theory for Engineering.



%% \section{Conference Publications}
%% %
%% Freuler, R.J., M.J.~Hoffmann, T.P.~Pavlic, J.M.~Beams, J.P.~Radigan,
%% P.K.~Dutta, J.T.~Demel, and E.D.~Justen. 2003. Experiences with a
%% Comprehensive Freshman Hands-On Course -- Designing, Building, and
%% Testing Small Autonomous Robots. Proceedings of the 2003
%% \href{http://www.asee.org/}{American Society for Engineering Education}
%% Annual Conference \& Exposition.



%% \section{Professional Experience}
%% %
%% \href{http://www.ni.com/}{\textbf{National Instruments}},
%% Austin, Texas USA
%% \begin{outerlist}

%% \item[] \textit{Hardware R\&D Intern for Multifunction DAQ}%
%%         \hfill \textbf{June 2003 to September 2003}
%% \begin{innerlist}
%% \item Designed final verification testing fixture for use with STC2 MIO
%%         products.
%% \item Designed and executed study of the effect of varying burn-in time
%%         on long-term drift of common industry voltage references.
%% \end{innerlist}

%% \item[] \textit{Hardware R\&D Intern for Multifunction DAQ}%
%%         \hfill \textbf{June 2002 to September 2002}
%% \begin{innerlist}
%% \item Designed and performed validation tests on new 16-bit 800 kHz
%%         NI-6120 SMIO DAQ board.

%% \item Designed high quality filter/amplifier source for use with NI-5411
%%         arbitrary function generator.
%% \end{innerlist}

%% \end{outerlist}

%% \section{Service}
%% %
%% Director of Computers,
%% \href{http://ec.osu.edu/}{Engineers' Council},
%% \href{http://www.osu.edu/}{The Ohio State University}, 2002

%% \blankline

%% \href{http://www.osufirst.org/}{OSU FIRST Robotics Team},
%% \href{http://www.osu.edu}{The Ohio State University}, 2000--2004
%% \begin{innerlist}
%% \item Introduced middle school and high school students to science and
%%         technology by participating with them in national robotics
%%         competitions.
%% \item Led 2002 team to regional silver medal
%%         \href{http://www.firstwiki.org/Engineering_Inspiration_Award}
%%              {\emph{Engineering Inspiration Award}}.
%% \item \emph{Lead Team Mentor}, 2002--2004
%% \item \emph{Component Design Team Lead Mentor}, 2001--2002
%% \end{innerlist}

%% \blankline

%% \href{http://www.linuxvirtualserver.org/}
%%      {Linux Virtual Server Project}, 1999--2000
%% \begin{innerlist}
%% \item Early member of the team that formed the open source project that
%%         is now an important load balancing solution for the Linux
%%         software platform.
%% \end{innerlist}

%% \blankline

%% \href{http://www.gcfn.org/}
%%      {Greater Columbus Free-Net}, 1995--1997
%% \begin{innerlist}
%% \item Provided technical support services.
%% \end{innerlist}

%% \blankline

%% CompuTeen Bulletin Board System, 1993--1995
%% \begin{innerlist}
%% \item Administrated dial-up bulletin board system.
%% \item Founded and administrated TeenLiNK, an international electronic
%%         mail network that spread through the United States, Canada, and
%%         Australia and delivered mail over a series of electronic dial-up
%%         drop offs.
%% \end{innerlist}


%% \section{Awards}
%% %
%% \href{http://www.nsf.gov/}{National Science Foundation}
%% \begin{innerlist}
%% \item \href{http://www.nsfgk12.org/}{GK-12 Fellowship}, 2006
%% \item \href{http://www.nsf.gov/grfp}
%%            {Graduate Research Fellowship} Honorable Mention, 2005
%% \end{innerlist}

%% \blankline

%% \href{http://www.osu.edu}{The Ohio State University}
%% \begin{innerlist}
%% \item \href{http://www.gradsch.osu.edu/Content.aspx?Content=44&itemid=2}
%%            {Dean's Distinguished University Fellowship}, 2004
%% \item Electrical and Computer Engineering Bradshaw Scholarship,
%%         2002--2004
%% \item Electrical and Computer Engineering Shafstall Scholarship,
%%         2001--2003
%% \item University Scholarship, 1999--2003
%% \end{innerlist}


\end{document}

%%%%%%%%%%%%%%%%%%%%%%%%%% End CV Document %%%%%%%%%%%%%%%%%%%%%%%%%%%%%
